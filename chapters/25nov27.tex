\documentclass[../main.tex]{subfiles}

\begin{document}

\chapter{范小青《牵手》}

曾明的眼睛不是一下子坏了的,他先是得了一种眼病,医生就预言曾明的眼睛不行了。最后曾明真的成了一个盲人。

在黑暗的世界中生存下去,这就是曾明必须选择的路。曾明被介绍到街道办的福利工厂去工作,往一块小小的金属板上辗螺丝。上班的时候,把一台收音机开着,节目很丰富,多半是直播形式的,盲人们常常放下手里的活,给电台打热线电话,曾明很快就被吸引,觉得生活有意味得多了。

辗螺丝对曾明来说,真是小菜一碟,进厂不多久,就已经很熟练。有一天曾明起身去方便,不小心和邻近的老陶撞了一下,金属板翻到了一处,他们一起蹲下来拣金属板。才拣了几十只,曾明就再也摸不着了,便有些急,道:“我做了一百只了,怎么只有这一点点?”

老陶随口回道:“这么凑巧哇,刚好一百。”

“我数到一百,才起身去上厕所。”

几个人笑了起来,曾明道:“你们的意思,是我瞎说?”

没人回答曾明的问题,曾明便起身找负责人,负责人听了,也是一笑,道:“算了,又不计件,不要计较了吧。”

“凡事总有个道理。”

“扯不上,工资又不挂钩……”

曾明再没有说话。晚上回家听电台节目,曾明打了一个热线电话,把事情说了,主持人告诉他,这算是一种病态心理,解除的最好办法,就是找人倾诉。主持人告诉曾明,曾明住处不远的另一条街上,有一位在街道办事处做调解的刘主任,建议曾明找那位刘主任说说,刘主任是调解战线的先进。

曾明在某一天果真找到那地方去,人们把他引到主任的办公室时,曾明听到刘主任正在调解民事纠纷,他听主任说得在情在理,很快就把当事双方说通了,高兴而去。曾明听到喝水的声音,接着刘主任问他:“你是不是我们这个街道的?”

“是,是电台的主持人叫我来的。”

“是小丁吧,他常常介绍人来我这里。”

曾明就把事情说了,说罢却有好一阵没有听到刘主任的声音,只觉得周围有一种沉静压抑的气氛,曾明还以为刘主任出去了呢,忍不住问道:“你在吗?”

刘主任说:“我在……”停顿一下,问道:“你是盲人?”

曾明心下有些奇怪,但并没有往深里想,只是点头道:“是的,得了一种奇怪的眼病,医不好。”

“这么说来,你失明的时间不很长?”在曾明的感觉中,刘主任的声音好像离得很远。

“半年吧。”

“你……”刘主任又停顿了一下,问道:“你失明以后,做梦吗?”曾明愣了一下,摇了摇头。

刘主任又问一遍:“你失明以后,做梦吗?”

“没,好像没有梦见过什么。”他不明白刘主任问他这个做什么,或许是一种心理治疗。

“盲人做梦,若能看见东西,古时候称作天眼开。”

曾明想了想,说:“那恐怕说的是先天的盲人吧,像我们这样,应该是能做梦的,人若盲了,已经够痛苦,若连梦也做不起来,那就更惨,不能这么不公平吧。”

“我想也是,只是盲人不做梦,这是事实呀。”

“你怎么知道?”

刘主任没有回答曾明的这个问题,却回到了曾明的主题上,说:“你心中的这股气,其实不是对着老陶的,你说是不是?从根本上说你对于自己的失明一直郁闷不平,看起来你已经适应了失明以后的生活,其实你并没有适应,你还需要继续适应……”

曾明打断刘主任的话:“没有失明的人,怎么能够体验失明人的滋味,就像你,怕是不能体谅我的心情吧。”

刘主任笑了一下,说:“也许吧……我问你一个问题,你说,在盲人中,是先天的盲人更痛苦呢,还是后天的失明更痛苦?”

曾明一时回答不出来。

刘主任说:“这个问题我总是想不明白,我总是在想……”下面的话被一阵人声打断,有人进来说道:“刘主任,又来人了。”

曾明知道刘主任有工作了,便站起来道:“刘主任,你忙,我先走了,过日我再过来就是。”

刘主任说:“好,我领你出去。”就有一只热乎乎的手伸过来。一路过来,刘主任没有和曾明说话,曾明再一次感受到在刘主任办公室里感觉到的那种沉静压抑。

曾明继续到福利厂上班,大家和他仍像以前一样亲切,好像谁都不记得曾经有过一丝不愉快的事情。一天夜里,曾明做了一个梦,梦见刘主任对他说:“你怎么不来了?我很想你。”醒来后,曾明的心里有些不宁。过了几天,他又到刘主任那里去,这一回曾明只让人把他引到走廊端头,他自己沿着走廊,很快摸到了刘主任的办公室,进去,刘主任说:“我已经听到了你的脚步声。”

曾明说:“你的耳朵真好。”

刘主任说:“你来得正好,今天是我的休息日,我们一起出去走走好吗?”

“到哪里?”

“到那地方你就知道。”

一只热乎乎的手伸过来,曾明的手被那只手牵着,他们一起往外走,以曾明的感觉,好像到了一个类似公园的地方。

“你听到了什么?”

“鸟叫,很多很多的鸟。”

刘主任笑了,说:“是的,他们都在这里遛鸟,今天比赛。”

曾明说:“比什么?”

“比鸟的叫声。”

在一片叽喳的鸟鸣声中,曾明突然感觉到自己内心一片明亮,刘主任的热乎乎的手又伸过来牵住了他的手,说:“走,我们上那边看看去。”

他们牵着手走了几步,曾明听到身边有人在说话,他们说,瞧,两个瞎子手牵着手在走路呢。

\rightline{(范小青《牵手》,有删改)}

\begin{mybox}
	4. 下列对小说相关内容 理解,正确的一项是( )

	\begin{enumerate}[label=(\Alph*)]
		\item 曾明主动给电台的主持人打热线电话,不仅是为了发发牢骚,实际上他已经意识到自己存在着严重的心理问题。

		\item 知道曾明是个盲人后,刘主任并没有改变自己调解纠纷时的惯常做法,这说明刘主任对人一视同仁,维护他人尊严。

		\item “是先天的盲人更痛苦呢,还是后天的失明更痛苦?”这个问题的提出与引发的思考,构成了小说的基调与主题。

		\item 相比于盲人生活的不便,小说更侧重于描写他们精神上面临的困惑,也体现出对残疾人心理问题的理解与关注。
	\end{enumerate}

	5. 下列对小说艺术特色的分析鉴赏,不正确的一项是( )

	\begin{enumerate}[label=(\Alph*)]
		\item 工人经常给电台打热线电话,写出了福利工厂相对宽松的工作环境,也为后面的故事埋下了伏笔。

		\item 小说通过曾明与老陶口角这个偶发事件,具体展现了“在黑暗的世界中生存下去”的现实问题,由小及大,构思自然。

		\item 第一次见刘主任时,曾明“觉得周围有一种沉静压抑 气氛”,这写出了盲人心理上的敏感。

		\item 小说语言平实、质朴、简洁,这种语言风格体现着作者对盲人世界的认识,看似平淡,实则很有韵味。
	\end{enumerate}

	6. 小说直至最后才交待刘主任是个盲人,但前文已有多处细节予以暗示,请找出相关细节。

	7. 小说从曾明的角度讲述故事,有怎样的艺术效果?请结合小说简要分析。
\end{mybox}


\end{document}
